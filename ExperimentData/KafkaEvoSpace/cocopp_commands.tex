\providecommand{\algorithmA}{kafka}
\providecommand{\algorithmB}{EvoSpace}
\providecommand{\algname}{EvoSpace{}}
\providecommand{\algfolder}{EvoSpace/}
\providecommand{\bbobecdfcaptionallgroups}[1]{
Empirical cumulative distribution of simulated (bootstrapped)
             runtimes, measured in number of objective function evaluations,
             divided by dimension (FEvals/DIM) for the $51$ 
             targets $10^{[-8..2]}$ for all function groups and all 
             dimensions. The aggregation over all 24 
             functions is shown in the last plot.
}
\providecommand{\bbobecdfcaptionsinglefcts}[2]{
Empirical cumulative distribution of simulated (bootstrapped) runtimes in number
             of objective function evaluations divided by dimension (FEvals/DIM) for the 
             $51$ targets $10^{[-8..2]}$
             for functions $f_{#1}$ to $f_{#2}$ and all dimensions. 
}
\providecommand{\bbobpptablecaption}[1]{
%
        Average running time (\aRT\ in number of function 
        evaluations) divided by the \aRT\ of the best algorithm from BBOB 2009 in #1. The \aRT\ 
        and in braces, as dispersion measure, the half difference between 90 and 
        10\%-tile of bootstrapped run lengths appear in the second row of each cell,  
        the best \aRT\
        %
        (preceded by the target \Df-value in \textit{italics}) in the first. 
        \#succ is the number of trials that reached the target value of the last column.
        %
        The median number of conducted function evaluations is additionally given in 
        \textit{italics}, if the target in the last column was never reached. 
        \textbf{Bold} entries are statistically significantly better (according to
        the rank-sum test) compared to the best algorithm from BBOB 2009, with
        $p = 0.05$ or $p = 10^{-k}$ when the number $k > 1$ is following the
        $\downarrow$ symbol, with Bonferroni correction by the number of
        functions.\cocoversion
        
}
\providecommand{\bbobppfigdimlegend}[1]{
%
        Scaling of the running time with dimension to reach certain target values \Df.
        Lines: average runtime (\aRT);
        Cross (+): median runtime of successful runs to reach the most difficult
        target that was reached at least once (but not always);
        Cross ({\color{red}$\times$}): maximum number of
        $f$-evaluations in any trial. Notched boxes: interquartile range with median of simulated runs. 
        All values are divided by dimension and  
        plotted as $\log_{10}$ values versus dimension. %
        %
        Shown is the \aRT\ for fixed values of $\Df = 10^k$ with $k$ given
        in the legend.
        Numbers above \aRT-symbols (if appearing) indicate the number of trials
        reaching the respective target. The light thick line with diamonds indicates the best algorithm from BBOB 2009 for the most difficult target.  Horizontal lines mean linear scaling, slanted
        grid lines depict quadratic scaling. Left: EvoSpace, Right: KafkEO.
}
\providecommand{\bbobpprldistrlegend}[1]{
%
         Empirical cumulative distribution functions (ECDF), plotting the fraction of
         trials with an outcome not larger than the respective value on the $x$-axis.
         #1%
         Left subplots: ECDF of the number of function evaluations (FEvals) divided by search space dimension $D$,
         to fall below $\fopt+\Df$ with $\Df=10^{k}$, where $k$ is the first value in the legend.
         The thick red line represents the most difficult target value $\fopt+ 10^{-8}$. %
         Legends indicate for each target the number of functions that were solved in at
         least one trial within the displayed budget.
         Right subplots: ECDF of the best achieved $\Df$
         for running times of $0.5D, 1.2D, 3D, 10D, 100D, 1000D,\dots$
         function evaluations
         (from right to left cycling cyan-magenta-black\dots) and final $\Df$-value (red),
         where \Df and \textsf{Df} denote the difference to the optimal function value. 
         Light brown lines in the background show ECDFs for the most difficult target of all
            algorithms benchmarked during BBOB-2009.
}
\providecommand{\bbobloglossfigurecaption}[1]{
%
        \aRT\ loss ratios (see Figure~\ref{tab:aRTloss} for details).

        Each cross ({\color{blue}$+$}) represents a single function, the line
        is the geometric mean.
        
}
\providecommand{\bbobloglosstablecaption}[1]{
%
        \aRT\ loss ratio versus the budget in number of $f$-evaluations
        divided by dimension.
        For each given budget \FEvals, the target value \ftarget\ is computed
        as the best target $f$-value reached within the
        budget by the given algorithm.
        Shown is then the \aRT\ to reach \ftarget\ for the given algorithm
        or the budget, if the best algorithm from BBOB 2009
        reached a better target within the budget,
        divided by the \aRT\ of the best algorithm from BBOB 2009 to reach \ftarget.
        Line: geometric mean. Box-Whisker error bar: 25-75\%-ile with median
        (box), 10-90\%-ile (caps), and minimum and maximum \aRT\ loss ratio
        (points). The vertical line gives the maximal number of function evaluations
        in a single trial in this function subset. See also
        Figure~\ref{fig:aRTlogloss} for results on each function subgroup.\cocoversion
        
}
\providecommand{\pptablefooter}{
\end{tabular}
}
\providecommand{\pptableheader}{
\begin{tabular}{@{}c@{}|*{7}{@{}r@{}@{}l@{}}|@{}r@{}@{}l@{}}
$\Delta f$ & \multicolumn{2}{c}{1e+1} & \multicolumn{2}{c}{1e+0} & \multicolumn{2}{c}{1e-1} & \multicolumn{2}{c}{1e-2} & \multicolumn{2}{c}{1e-3} & \multicolumn{2}{c}{1e-5} & \multicolumn{2}{c}{1e-7} & \multicolumn{2}{|@{}r@{}}{\#succ}\\\hline
}
\providecommand{\algname}{kafka{}}
\providecommand{\algfolder}{kafka/}
\providecommand{\bbobecdfcaptionallgroups}[1]{
Empirical cumulative distribution of simulated (bootstrapped)
             runtimes, measured in number of objective function evaluations,
             divided by dimension (FEvals/DIM) for the $51$ 
             targets $10^{[-8..2]}$ for all function groups and all 
             dimensions. The aggregation over all 24 
             functions is shown in the last plot.
}
\providecommand{\bbobecdfcaptionsinglefcts}[2]{
Empirical cumulative distribution of simulated (bootstrapped) runtimes in number
             of objective function evaluations divided by dimension (FEvals/DIM) for the 
             $51$ targets $10^{[-8..2]}$
             for functions $f_{#1}$ to $f_{#2}$ and all dimensions. 
}
\providecommand{\bbobpptablecaption}[1]{
%
        Average running time (\aRT\ in number of function 
        evaluations) divided by the \aRT\ of the best algorithm from BBOB 2009 in #1. The \aRT\ 
        and in braces, as dispersion measure, the half difference between 90 and 
        10\%-tile of bootstrapped run lengths appear in the second row of each cell,  
        the best \aRT\
        %
        (preceded by the target \Df-value in \textit{italics}) in the first. 
        \#succ is the number of trials that reached the target value of the last column.
        %
        The median number of conducted function evaluations is additionally given in 
        \textit{italics}, if the target in the last column was never reached. 
        \textbf{Bold} entries are statistically significantly better (according to
        the rank-sum test) compared to the best algorithm from BBOB 2009, with
        $p = 0.05$ or $p = 10^{-k}$ when the number $k > 1$ is following the
        $\downarrow$ symbol, with Bonferroni correction by the number of
        functions.\cocoversion
        
}
\providecommand{\bbobppfigdimlegend}[1]{
%
        Scaling of runtime with dimension to reach certain target values \Df.
        Lines: average runtime (\aRT);
        Cross (+): median runtime of successful runs to reach the most difficult
        target that was reached at least once (but not always);
        Cross ({\color{red}$\times$}): maximum number of
        $f$-evaluations in any trial. Notched boxes: interquartile range with median of simulated runs; 
        All values are divided by dimension and  
        plotted as $\log_{10}$ values versus dimension. %
        %
        Shown is the \aRT\ for fixed values of $\Df = 10^k$ with $k$ given
        in the legend.
        Numbers above \aRT-symbols (if appearing) indicate the number of trials
        reaching the respective target. The light thick line with diamonds indicates the best algorithm from BBOB 2009 for the most difficult target.  Horizontal lines mean linear scaling, slanted
        grid lines depict quadratic scaling.
}
\providecommand{\bbobpprldistrlegend}[1]{
%
         Empirical cumulative distribution functions (ECDF), plotting the fraction of
         trials with an outcome not larger than the respective value on the $x$-axis.
         #1%
         Left subplots: ECDF of the number of function evaluations (FEvals) divided by search space dimension $D$,
         to fall below $\fopt+\Df$ with $\Df=10^{k}$, where $k$ is the first value in the legend.
         The thick red line represents the most difficult target value $\fopt+ 10^{-8}$. %
         Legends indicate for each target the number of functions that were solved in at
         least one trial within the displayed budget.
         Right subplots: ECDF of the best achieved $\Df$
         for running times of $0.5D, 1.2D, 3D, 10D, 100D, 1000D,\dots$
         function evaluations
         (from right to left cycling cyan-magenta-black\dots) and final $\Df$-value (red),
         where \Df and \textsf{Df} denote the difference to the optimal function value. 
         Light brown lines in the background show ECDFs for the most difficult target of all
            algorithms benchmarked during BBOB-2009.
}
\providecommand{\bbobloglossfigurecaption}[1]{
%
        \aRT\ loss ratios (see Figure~\ref{tab:aRTloss} for details).

        Each cross ({\color{blue}$+$}) represents a single function, the line
        is the geometric mean.
        
}
\providecommand{\bbobloglosstablecaption}[1]{
%
        \aRT\ loss ratio versus the budget in number of $f$-evaluations
        divided by dimension.
        For each given budget \FEvals, the target value \ftarget\ is computed
        as the best target $f$-value reached within the
        budget by the given algorithm.
        Shown is then the \aRT\ to reach \ftarget\ for the given algorithm
        or the budget, if the best algorithm from BBOB 2009
        reached a better target within the budget,
        divided by the \aRT\ of the best algorithm from BBOB 2009 to reach \ftarget.
        Line: geometric mean. Box-Whisker error bar: 25-75\%-ile with median
        (box), 10-90\%-ile (caps), and minimum and maximum \aRT\ loss ratio
        (points). The vertical line gives the maximal number of function evaluations
        in a single trial in this function subset. See also
        Figure~\ref{fig:aRTlogloss} for results on each function subgroup.\cocoversion
        
}
\providecommand{\pptablefooter}{
\end{tabular}
}
\providecommand{\pptableheader}{
\begin{tabular}{@{}c@{}|*{7}{@{}r@{}@{}l@{}}|@{}r@{}@{}l@{}}
$\Delta f$ & \multicolumn{2}{c}{1e+1} & \multicolumn{2}{c}{1e+0} & \multicolumn{2}{c}{1e-1} & \multicolumn{2}{c}{1e-2} & \multicolumn{2}{c}{1e-3} & \multicolumn{2}{c}{1e-5} & \multicolumn{2}{c}{1e-7} & \multicolumn{2}{|@{}r@{}}{\#succ}\\\hline
}
